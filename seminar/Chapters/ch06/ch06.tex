\chapter{CONCLUSION}
Comparative analysis between different word embedding model is done in this seminar report. The goal of the task is to evaluate system for the Nepali news classification. So, this report evaluated the different word representation methods in the framework of a controlled task. Nepali is a complex language with a wide range of vocabulary containing many rare words.
Language structure use of complex words, multiple meaning in different context all these reasons make it difficult to choose one model as the best for Nepali word classification. Additionally, the content of the dataset also plays the vital role in deciding this.\\
The two word embedding techniques, Word2Vec and ELMO as well as two machine learning algorithms namely SVM and Random Forest are used for the comparative study for the Nepali text classification. Additionally, the dataset for this study is managed by web scraping process. As this report evaluation shows, the model with ELMO vector slightly caught up and surpassed the model with Word2Vec, obtaining higher test accuracy.\\
It is also observed that using Random Forest to classify news led to the high-test accuracy as well as precision and recall as compared to the classification report generated by the Support Vector Machine learning algorithm. However, this model gives the slight balanced classification report.\\
Contrary to the expectation to this seminar, contextualized ELMo vectors did not result in notable performance gain compared to regular Word2Vec model. This could be in part since the stemming with the Nepali text is not easily done in this time. However, increasing the size of dataset and the big overhead in training and testing time for the ELMo model could improve the results.



